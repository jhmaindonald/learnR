% !Rnw root = learnR.Rnw




\marginnote{Important R technical terms include {\em object}, {\em
    {\em workspace}, working directory}, {\em image file}, {\em
    package}, {\em library}, {\em database} and {\em search list}.}

\noindent
\fbox{\parbox{\textwidth}{
\vspace*{-2pt}

\begin{tabular}{ll}
  Object & Objects can be data objects, function objects,\\
         & formula objects, expression objects, \ldots\\
 &  Use \txtt{ls()} to list contents of current workspace.\\[6pt]
Workspace & User's ``database'', where the user can make\\
& additions or changes or deletions.\\[6pt]
 Working  & Default directory for reading or writing files.\\
directory & Use a new working directories for a new project.\\[6pt]
 Image files & Use to store R objects, e.g., workspace contents.\\
  & (The expected file extension is \textbf{.RData} or \textbf{.rda})\\[6pt]
Search list & \txtt{search()} lists `databases' that R will
search.\\
 & \txtt{library()} adds packages to the search list\\
\end{tabular}
}}
\marginnote[-69pt]{Use the relevant menu. or enter \margtt{save.image()}
on the command line, to store or back up workspace contents.
During a long R session, do frequent saves!}

\section{The Working Directory and the Workspace}

Each R session has a \textit{working directory} and a workspace.  If
not otherwise instructed, R looks in the \textit{working directory}
for files, and saves files that are output to it.

\marginnote[12pt]{The workspace is a {\em volatile} database
that, unless saved, will disappear at the end of the session.}
The {\em workspace} is at the base of a list of search locations,
known as {\em databases}, where R will as needed search for objects.
It holds objects that the user has created or input, or that were
there at the start of the session and not later removed.

The workspace changes as objects are added or deleted or modified.
Upon quitting from R (type \txtt{q()}, or use the relevant menu item),
users are asked whether they wish to save the current workspace.
\marginnote{The file \txtt{.RData} has the name {\em image} file.
From it the workspace can, as and when required, be reconstructed.}
If saved, it is stored in the file {\bf .RData}, in the current
working directory.  When an R session is next started in that
working directory, the off-the-shelf action is to look for a file
named {\bf .RData}, and if found to reload it.

\subsection*{Setting the Working Directory}
When a session is started by clicking on a Windows icon, the icon's
Properties specify the \underline{Start In} directory.\footnote{When a
  Unix or Linux command starts a session, the default is to use the
  current directory.} Type \txtt{getwd()} to identify the current
working directory.

It is good practice to use a separate working directory, and
associated workspace or workspaces, for each different project. On
Windows systems, copy an existing R icon, rename it as desired, and
change the \underline{Start In} directory to the new working
directory.  The working directory can be changed\sidenote{To make a
  complete change to a new workspace, first save
  the existing workspace, and type \margtt{rm(list=ls(all=TRUE)} to
  empty its contents. Then change the working directory and
  load the new workspace.} once a session has started, either from the
menu (if available) or from the command line.  Changing the working
directory from within a session requires a clear head; it is usually
best to save one's work, quit, and start a new session.

\section{Work and Data Maintenance}

\subsection{Maintenance of R scripts}

\marginnote[12pt]{Note again RStudio's abilities for managing and
  keeping R scripts.}  It is good practice to maintain a transcript
from which work done during the session, including data input and
manipulation, can as necessary be reproduced.  Where calculations are
quickly completed, this can be re-executed when a new session is
started, to get to the point where the previous session left off.

\subsection{Saving and retrieving R objects}\label{ss:saveobjs}

Use \txtt{save()} to save one or more named objects\marginnote{The command \txtt{save.image()}) saves everything in the workspace, by
default into a file named \textbf{.RData} in the working directory.
Or, from a GUI interface, click on the relevant menu item.}
  into an image file.  Use \txtt{load()} to reload the image
file contents back into the workspace.  The following demonstrate the
explicit use of \txtt{save()} and \txtt{load()} commands:
\begin{knitrout}
\definecolor{shadecolor}{rgb}{0.969, 0.969, 0.969}\color{fgcolor}\begin{kframe}
\begin{alltt}
\hlstd{volume} \hlkwb{<-} \hlkwd{c}\hlstd{(}\hlnum{351}\hlstd{,} \hlnum{955}\hlstd{,} \hlnum{662}\hlstd{,} \hlnum{1203}\hlstd{,} \hlnum{557}\hlstd{,} \hlnum{460}\hlstd{)}
\hlstd{weight} \hlkwb{<-} \hlkwd{c}\hlstd{(}\hlnum{250}\hlstd{,} \hlnum{840}\hlstd{,} \hlnum{550}\hlstd{,} \hlnum{1360}\hlstd{,} \hlnum{640}\hlstd{,} \hlnum{420}\hlstd{)}
\hlkwd{save}\hlstd{(volume, weight,} \hlkwc{file}\hlstd{=}\hlstr{"books.RData"}\hlstd{)}
  \hlcom{# Can save many objects in the same file}
\hlkwd{rm}\hlstd{(volume, weight)}      \hlcom{# Remove volume and weight}
\hlkwd{load}\hlstd{(}\hlstr{"books.RData"}\hlstd{)}     \hlcom{# Recover the saved objects}
\end{alltt}
\end{kframe}
\end{knitrout}
\marginnote[-36pt]{See Subsection \ref{ss:moreattach} for
use of \margtt{attach("books.RData")} in place of
\margtt{load("books.RData")}.}

Where it will be time-consuming to recreate objects in the workspace, 
users will be advised to save (back up) the current workspace image from time to time, e.g., into a file, preferably with a suitably
mnemonic name.  For example:
\begin{knitrout}
\definecolor{shadecolor}{rgb}{0.969, 0.969, 0.969}\color{fgcolor}\begin{kframe}
\begin{alltt}
\hlstd{fnam} \hlkwb{<-} \hlstr{"2014Feb1.RData"}
\hlkwd{save.image}\hlstd{(}\hlkwc{file}\hlstd{=fnam)}
\end{alltt}
\end{kframe}
\end{knitrout}
\marginnote[-24pt]{Before saving the workspace, consider use of
\txtt{rm()} to remove objects that are no longer required.}

Two further possibilities are:
\begin{itemizz}
\item[-] Use \txtt{dump()} to save one or more objects in a text
format. For example:
\begin{knitrout}
\definecolor{shadecolor}{rgb}{0.969, 0.969, 0.969}\color{fgcolor}\begin{kframe}
\begin{alltt}
\hlstd{volume} \hlkwb{<-} \hlkwd{c}\hlstd{(}\hlnum{351}\hlstd{,} \hlnum{955}\hlstd{,} \hlnum{662}\hlstd{,} \hlnum{1203}\hlstd{,} \hlnum{557}\hlstd{,} \hlnum{460}\hlstd{)}
\hlstd{weight} \hlkwb{<-} \hlkwd{c}\hlstd{(}\hlnum{250}\hlstd{,} \hlnum{840}\hlstd{,} \hlnum{550}\hlstd{,} \hlnum{1360}\hlstd{,} \hlnum{640}\hlstd{,} \hlnum{420}\hlstd{)}
\hlkwd{dump}\hlstd{(}\hlkwd{c}\hlstd{(}\hlstr{"volume"}\hlstd{,} \hlstr{"weight"}\hlstd{),} \hlkwc{file}\hlstd{=}\hlstr{"volwt.R"}\hlstd{)}
\hlkwd{rm}\hlstd{(volume, weight)}
\hlkwd{source}\hlstd{(}\hlstr{"volwt.R"}\hlstd{)}      \hlcom{# Retrieve volume & weight}
\end{alltt}
\end{kframe}
\end{knitrout}
\vspace*{-8pt}
\item[-] Use \txtt{write.table()} to write a data frame to a text file.
\end{itemizz}


\section{Packages and System Setup}
\marginnote{For download or installation of R or CRAN packages, use
  for preference a local mirror.  In Australia {\scriptsize
    \url{http://cran.csiro.au}} is a good choice.  The mirror can be
  set from the Windows or Mac GUI. Alternatively (on any system), type
  \txtt{chooseCRANmirror()} and choose from the list that pops up.}
\noindent \fbox{\parbox{\textwidth}{
\vspace*{-3pt}

\begin{tabular}{ll}
  Packages & Packages are structured collections of R\\
  & functions and/or data and/or other objects.\\[6pt]
Installation & R Binaries include 'recommended' packages.\\
of packages & Install other packages, as required,\\[6pt]
\txtt{library()} & Use to attach a package, e.g., \txtt{library(DAAG)} \\
 & Once attached, a package is added to the list of \\
 & ``databases'' that R searches for objects.
\end{tabular}
}
}
\vspace*{8pt}

\noindent
An R installation is structured as a library of packages.
\begin{itemizz}
\item All installations should have the base packages (one of them is
  called \textit{base}).  These provide the infrastructure for other
  packages.
\item Binaries that are available from CRAN sites include, also, all
the recommended packages.
\item Other packages can be installed as required, from a CRAN mirror
site, or from another repository.
\end{itemizz}

\begin{marginfigure}[12pt]
To discover which packages are attached, enter one of:
\begin{knitrout}
\definecolor{shadecolor}{rgb}{0.969, 0.969, 0.969}\color{fgcolor}\begin{kframe}
\begin{alltt}
\hlkwd{search}\hlstd{()}
\hlkwd{sessionInfo}\hlstd{()}
\end{alltt}
\end{kframe}
\end{knitrout}
Use \margtt{sessionInfo()} to get more detailed information.
\end{marginfigure}
A number of packages are by default attached
at the start of a session.  To attach other packages, use
\txtt{library()} as required.

\subsection{Installation of R packages}\label{ss:installpack}
Section \ref{sec:pkgs} described the installation of packages from the
internet. Note also the use of \txtt{update.packages()} or its
equivalent from the GUI menu.  This identifies packages for which
updates are available, offering the user the option to proceed with
the update.

\marginnote[12pt]{Arguments are a vector of package names and a destination
  directory \margtt{destdir} where the latest file versions will be
  saved as {\bf .zip} or (MacOS X) {\bf.tar.gz} files.}
The function \txtt{download.packages()} allows the downloading of
packages for later installation.  The menu, or
\margtt{install.packages()}, can then be used to install the packages
from the local directory.  For command line installation of packages that
are in a local directory, call \txtt{install.packages()} with
\txtt{pkgs} giving the files (with path, if necessary), and with the
argument \txtt{repos=NULL}.

\marginnote[12pt]{On Unix and Linux systems, gzipped tar files
  can be installed using the shell command:\\[3pt]
\: R CMD INSTALL xx.tar.gz\\[3pt]
\noindent (replace xx.tar.gz by the file name.)}
If for example the binary
\textbf{DAAG\_1.22.zip} has been downloaded to
\textbf{D:$\boldsymbol{\backslash}$tmp$\boldsymbol{\backslash}$}, it
can be installed thus
\begin{knitrout}
\definecolor{shadecolor}{rgb}{0.969, 0.969, 0.969}\color{fgcolor}\begin{kframe}
\begin{alltt}
\hlkwd{install.packages}\hlstd{(}\hlkwc{pkgs}\hlstd{=}\hlstr{"D:/DAAG_1.22.zip"}\hlstd{,}
                 \hlkwc{repos}\hlstd{=}\hlkwa{NULL}\hlstd{)}
\end{alltt}
\end{kframe}
\end{knitrout}
On the R command line, be sure to replace the usual Windows backslashes
by forward slashes.

Use \txtt{.path.package()} to get the path of a currently attached package
(by default for all attached packages).

\subsection{The search path: library() and attach()}\label{ss:moreattach}
The R system maintains a {\em search path} (a list of {\em databases})
that determines, unless otherwise specified, where and in what order
to look for objects.  \marginnote{Database 1, where R looks first, is
  the user workspace, called \margtt{".GlobalEnv"}.}  The search list
includes the workspace, attached packages, and a so-called
\txtt{Autoloads} database. It may include other items also.

To get a snapshot of the search path,\marginnote{Packages other than
  {\em MASS} were attached at startup.} here taken after starting up
and entering \txtt{library(MASS)}, type:\marginnote[24pt]{If the
  process runs from RStudio, \margtt{"tools:rstudio"} will appear in
  place of \margtt{"tools:RGUI"}.}
\begin{Schunk}
\begin{Sinput}
search()
\end{Sinput}
\begin{Soutput}
 [1] ".GlobalEnv"        "package:MASS"
 [3] "tools:RGUI"        "package:stats"
 [5] "package:graphics"  "package:grDevices"
 [7] "package:utils"     "package:datasets"
 [9] "package:methods"   "Autoloads"
[11] "package:base"
\end{Soutput}
\end{Schunk}

For more detailed information that has version numbers of any packages
that are additional to base packages, type:
\begin{knitrout}
\definecolor{shadecolor}{rgb}{0.969, 0.969, 0.969}\color{fgcolor}\begin{kframe}
\begin{alltt}
\hlkwd{sessionInfo}\hlstd{()}
\end{alltt}
\end{kframe}
\end{knitrout}

\subsection*{The '\txtt{::}' notation}
Use notation such as \txtt{base::print()} to specify the package
where a function or other object is to be found.  This avoids any
risk of ambiguity when two or more objects with the same name can
be found in the search path.

In Subsection \ref{ss:layer} the notation \txtt{latticeExtra::layer()}
will be used to indicate that the function \txtt{layer()} from
the {\em latticeExtra} package is required, distinguishing it
from the function \txtt{layer()} in the {\em ggplot2} package.
\marginnote{It is necessary that the {\em latticeExtra} package
has been installed!}
Use of the notation \txtt{latticeExtra::layer()} makes unnecessary
prior use of \txtt{library(latticeExtra)} or its equivalent.

\subsection*{Attachment of image files}
\marginnote{Objects that are attached, whether workspaces or
packages (using \txtt{library()}) or other entities, are added
to the search list.}
The following adds the image file \txtt{books.RData} to the search list:
\begin{knitrout}
\definecolor{shadecolor}{rgb}{0.969, 0.969, 0.969}\color{fgcolor}\begin{kframe}
\begin{alltt}
\hlkwd{attach}\hlstd{(}\hlstr{"books.RData"}\hlstd{)}
\end{alltt}
\end{kframe}
\end{knitrout}
\noindent
The session then has access to objects in the file
\textbf{books.RData}.\marginnote{The file becomes a further
  ``database'' on the search list, separate from the workspace.}
  Note that if an object from the image file is modified,
the modified copy becomes part of the workspace.

In order to detach \txtt{books.RData}, proceed
thus:\marginnote{Alternatively, supply the numeric position of
  \txtt{books.RData} on the search list (if in position 2, then 2) as
  an argument to \margtt{detach()}.}
\begin{knitrout}
\definecolor{shadecolor}{rgb}{0.969, 0.969, 0.969}\color{fgcolor}\begin{kframe}
\begin{alltt}
\hlkwd{detach}\hlstd{(}\hlstr{"file:books.RData"}\hlstd{)}
\end{alltt}
\end{kframe}
\end{knitrout}

\subsection{$^*$Where does the R system keep its files?}

\marginnote{Note that R expects (and displays) either a single
  forward slash or double backslashes, where Windows would show a
  single backslash.}

Type \txtt{R.home()} to see where the R system has stored its files.
\begin{knitrout}
\definecolor{shadecolor}{rgb}{0.969, 0.969, 0.969}\color{fgcolor}\begin{kframe}
\begin{alltt}
\hlkwd{R.home}\hlstd{()}
\end{alltt}
\begin{verbatim}
[1] "/Library/Frameworks/R.framework/Resources"
\end{verbatim}
\end{kframe}
\end{knitrout}
Notice that the path appears in abbreviated form.  Type
\txtt{normalizePath(R.home())} to get the more intelligible result\\
\txtt{[1] "C:\textbackslash\textbackslash Program
  Files\textbackslash\textbackslash R\textbackslash\textbackslash R-2.15.2"}

By default, the command \txtt{system.file()} gives the path to the
base package.  For other packages, type, e.g.

\begin{fullwidth}

\begin{knitrout}
\definecolor{shadecolor}{rgb}{0.969, 0.969, 0.969}\color{fgcolor}\begin{kframe}
\begin{alltt}
\hlkwd{system.file}\hlstd{(}\hlkwc{package}\hlstd{=}\hlstr{"DAAG"}\hlstd{)}
\end{alltt}
\begin{verbatim}
[1] "/Users/johnm1/Library/R/3.4/library/DAAG"
\end{verbatim}
\end{kframe}
\end{knitrout}

\end{fullwidth}

To get the path to a file {\bf viewtemps.RData} that is stored with
the {\em DAAG} package in the {\bf misc} subdirectory, type:
\begin{fullwidth}
\begin{knitrout}
\definecolor{shadecolor}{rgb}{0.969, 0.969, 0.969}\color{fgcolor}\begin{kframe}
\begin{alltt}
\hlkwd{system.file}\hlstd{(}\hlstr{"misc/viewtemps.RData"}\hlstd{,} \hlkwc{package}\hlstd{=}\hlstr{"DAAG"}\hlstd{)}
\end{alltt}
\end{kframe}
\end{knitrout}
\end{fullwidth}

\subsection{Option Settings}

\begin{marginfigure}[44pt]
To display the setting for the
line width (in characters), type:
\begin{knitrout}
\definecolor{shadecolor}{rgb}{0.969, 0.969, 0.969}\color{fgcolor}\begin{kframe}
\begin{alltt}
\hlkwd{options}\hlstd{()}\hlopt{$}\hlstd{width}
\end{alltt}
\begin{verbatim}
[1] 54
\end{verbatim}
\end{kframe}
\end{knitrout}
\end{marginfigure}
Type \margtt{help(options)} to get full details of option settings.
There are a large number.  To change to 60 the number of characters
that will be printed on the command line, before wrapping, do:
\begin{knitrout}
\definecolor{shadecolor}{rgb}{0.969, 0.969, 0.969}\color{fgcolor}\begin{kframe}
\begin{alltt}
\hlkwd{options}\hlstd{(}\hlkwc{width}\hlstd{=}\hlnum{60}\hlstd{)}
\end{alltt}
\end{kframe}
\end{knitrout}

The printed result of calculations will, unless the default is changed
(as has been done for most of the output in this document) often
show more significant digits of output than are useful.  The following
demonstrates a global (until further notice) change:

\marginnote{Use \txtt{signif()} to affect one statement
  only. For example\\
\margtt{signif(sqrt(10),2)}\newline
\noindent
NB also the function \margtt{round()}.
}
\begin{knitrout}
\definecolor{shadecolor}{rgb}{0.969, 0.969, 0.969}\color{fgcolor}\begin{kframe}
\begin{alltt}
\hlkwd{sqrt}\hlstd{(}\hlnum{10}\hlstd{)}
\end{alltt}
\begin{verbatim}
[1] 3.162
\end{verbatim}
\begin{alltt}
\hlstd{opt} \hlkwb{<-} \hlkwd{options}\hlstd{(}\hlkwc{digits}\hlstd{=}\hlnum{2}\hlstd{)} \hlcom{# Change until further notice,}
                         \hlcom{# or until end of session.}
\hlkwd{sqrt}\hlstd{(}\hlnum{10}\hlstd{)}
\end{alltt}
\begin{verbatim}
[1] 3.2
\end{verbatim}
\begin{alltt}
\hlkwd{options}\hlstd{(opt)}             \hlcom{# Return to earlier setting}
\end{alltt}
\end{kframe}
\end{knitrout}
Note that \txtt{options(digits=2)} expresses a wish, which
R will not always obey!

\subsection*{Rounding will sometimes introduce small inconsistencies!}

For example:
\begin{knitrout}
\definecolor{shadecolor}{rgb}{0.969, 0.969, 0.969}\color{fgcolor}\begin{kframe}
\begin{alltt}
\hlkwd{round}\hlstd{(}\hlkwd{sqrt}\hlstd{(}\hlnum{85}\hlopt{/}\hlnum{7}\hlstd{),} \hlnum{2}\hlstd{)}
\end{alltt}
\begin{verbatim}
[1] 3.48
\end{verbatim}
\begin{alltt}
\hlkwd{round}\hlstd{(}\hlkwd{c}\hlstd{(}\hlkwd{sqrt}\hlstd{(}\hlnum{85}\hlopt{/}\hlnum{7}\hlstd{)}\hlopt{*}\hlnum{9}\hlstd{,}  \hlnum{3.48}\hlopt{*}\hlnum{9}\hlstd{),} \hlnum{2}\hlstd{)}
\end{alltt}
\begin{verbatim}
[1] 31.36 31.32
\end{verbatim}
\end{kframe}
\end{knitrout}

\section{Summary and Exercises}

\subsection{Summary}
\begin{itemizz}
\item[] Each R session has a working directory, where R will by
  default look for files or store files that are external to R.
\item[]
\marginnote{From within functions, R will look first in the
functions environment, and then if necessary look within the
search list.}
User-created R objects are added to the workspace, which is
at the base of a search list, i.e., a list of ``databases'' that R
will search when it looks for objects.
\item[] It is good practice to keep a separate workspace and
  associated working directory for each major project.  Use script
  files to keep a record of work.  \marginnote{Before making big
    changes to the workspace, it may be wise to save the
    existing workspace under a name (e.g., \margtt{Aug27.RData})
    different from the default \margtt{.RData}.}
\item[] At the end of a session an image of the workspace will
  typically (respond ``y'' when asked) be saved into the working
  directory.
\item[] Note also the use of \txtt{attach()} to give access to objects
  in an image (\textbf{.RData} or \textbf{.rda})
  file.\sidenote{Include the name of the file (optionally preceded by
    a path) in quotes.}
\item[] R has an extensive help system.  Use it!
\end{itemizz}

\subsection{Exercises}\label{ss:wd}
\marginnote{The function \margtt{DAAG::datafile()} is able to
place in the working directory any of the files: {\bf fuel.txt}
{\bf molclock1.txt}, {\bf molclock2.txt},  {\bf travelbooks.txt}.
Specify, e.g.\\
\margtt{datafile(file="fuel")}}
Data files used in these exercises are available from the web
page  \url{http://www.maths.anu.edu.au/~johnm/datasets/text/}.

\begin{enumerate}
\item
Place the file \textbf{fuel.txt} to your working directory.
\item Use \texttt{file.show()} to examine the file, or click on the
  RStudio \underline{Files} menu and then on the file name to display
  it.  Check carefully whether there is a header line.  Use the
  RStudio menu to input the data into R, with the name \texttt{fuel}.
  Then, as an alternative, use \texttt{read.table()} directly.  (If
  necessary use the code generated by RStudio as a crib.)  In each
  case, display the data frame and check that data have been input
  correctly.
\item \marginnote{A shortcut for placing these files in the working
    directory is:\\
    \margtt{datafile(file=c("molclock1",}\\
    \hfill \margtt{"molclock2"))}} Place the files
  \textbf{molclock1.txt} and \textbf{molclock2.txt} in a directory
  from which you can read them into R.  As in Exercise 1, use the
  RStudio menu to input each of these, then using
  \texttt{read.table()} directly to achieve the same result.  Check,
  in each case, that data have been input correctly.

  Use the function \txtt{save()} to save \txtt{molclock1}, into an R
  image file.  Delete the data frame \txtt{molclock1}, and check that
  you can recover the data by loading the image file.\label{ex:mol1}
\item The following counts, for each species, the number of missing values
for the column \texttt{root} of the data frame \texttt{DAAG::rainforest}:
\begin{knitrout}
\definecolor{shadecolor}{rgb}{0.969, 0.969, 0.969}\color{fgcolor}\begin{kframe}
\begin{alltt}
\hlkwd{library}\hlstd{(DAAG)}
\hlkwd{with}\hlstd{(rainforest,} \hlkwd{table}\hlstd{(}\hlkwd{complete.cases}\hlstd{(root), species))}
\end{alltt}
\end{kframe}
\end{knitrout}
For each species, how many rows are ``complete'', i.e., have no values
that are missing?
\item For each column of the data frame \texttt{MASS::Pima.tr2},
determine the number of missing values.
\item The function \texttt{dim()} returns the dimensions (a vector that
 has the number of rows, then number of columns) of data frames and
 matrices.  Use this function to find the number of rows in the data
 frames \texttt{tinting}, \texttt{possum} and \texttt{possumsites}
 (all in the \textit{DAAG} package).
\item Use \texttt{mean()} and \texttt{range()} to find the
mean and range of:
\begin{itemize}
  \item[(a)] the numbers 1, 2, \ldots, 21
  \item[(b)] the sample of 50 random normal values, that can be generated
    from a normaL distribution with mean 0 and variance 1 using the
    assignment \texttt{y <- rnorm(50)}.
  \item[(c)]
\marginnote{The \textit{datasets} package
    that has the data frame \margtt{women} is by default attached when R is
    started.}
the columns \texttt{height} and \texttt{weight} in the
    data frame \texttt{women}.
\end{itemize}
Repeat (b) several times, on each occasion generating a nwe set of
50 random numbers.
\item Repeat exercise 6, now applying the functions \texttt{median()} and
\texttt{sum()}.
\item Extract the following subsets from the data frame \texttt{DAAG::ais}
\begin{itemize}
\item[(a)] Extract the data for the rowers.
\item[(b)] Extract the data for the rowers, the netballers and the
tennis players.
\item[(c)] Extract the data for the female basketballers and rowers.
\end{itemize}
\item Use \texttt{head()} to check the names of the columns, and the first
few rows of data, in the data frame \texttt{DAAG::rainforest}.
Use \verb!table(rainforest$species)! to check the names and numbers of
each species that are present in the data.
The following extracts the rows for the species \textit{Acmena smithii}
\begin{knitrout}
\definecolor{shadecolor}{rgb}{0.969, 0.969, 0.969}\color{fgcolor}\begin{kframe}
\begin{alltt}
\hlstd{Acmena} \hlkwb{<-} \hlkwd{subset}\hlstd{(rainforest, species}\hlopt{==}\hlstr{"Acmena smithii"}\hlstd{)}
\end{alltt}
\end{kframe}
\end{knitrout}
The following extracts the rows for the species \texttt{Acacia mabellae} and
\texttt{Acmena smithii}:
\begin{fullwidth}

\begin{knitrout}
\definecolor{shadecolor}{rgb}{0.969, 0.969, 0.969}\color{fgcolor}\begin{kframe}
\begin{alltt}
\hlstd{AcSpecies} \hlkwb{<-} \hlkwd{subset}\hlstd{(rainforest, species} \hlopt \hlkwd{c}\hlstd{(}\hlstr{"Acacia mabellae"}\hlstd{,}
                                               \hlstr{"Acmena smithii"}\hlstd{))}
\end{alltt}
\end{kframe}
\end{knitrout}

\end{fullwidth}
Now extract the rows for all species except \texttt{C. fraseri}.
\end{enumerate}
\cleartooddpage
