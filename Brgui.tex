% !Rnw root = learnR.Rnw

%% ---preamble.tex----%% 
%% maxwidth is the original width if it is less than linewidth
%% otherwise use linewidth (to make sure the graphics do not exceed the margin)
\makeatletter
\def\maxwidth{ %
  \ifdim\Gin@nat@width>\linewidth
    \linewidth
  \else
    \Gin@nat@width
  \fi
}
\makeatother

\definecolor{fgcolor}{rgb}{0.345, 0.345, 0.345}

\definecolor{shadecolor}{rgb}{.97, .97, .97}
\definecolor{messagecolor}{rgb}{0, 0, 0}
\definecolor{warningcolor}{rgb}{1, 0, 1}
\definecolor{errorcolor}{rgb}{1, 0, 0}






To start the R commander, start R and enter:
\begin{Schunk}
\begin{Sinput}
library(Rcmdr)
\end{Sinput}
\end{Schunk}
This opens an R Commander script window, with the output window
underneath.  \marginnote[-1.5cm]{At startup, the R Commander checks
  whether all packages are available that are needed for the full
  range of features. If some are missing, the R commander offers to
  install them. (This requires a live internet connection.)}
This window can be closed by clicking on the {\large
  \texttt{$\times$}} in the top left corner. If thus closed,
enter \code{Commander()} to reopen it again later in the session.

\paragraph{From GUI to writing code:}
\marginnote[10pt]{The code can be run either from the
  script window or from the R console window (if open).}  The R commander
displays the code that it generates.  Users can take this code, modify
it, and re-run it.

\paragraph{The active data set:} There is, at any one time,
a single `active' data set. Start by clicking on the
\underline{Data} drop-down menu. To select
or create or change the active data set, do one of the following:
\begin{itemize}
\item Click on \underline{Active data set}, and pick from among data
  sets, if any, in the workspace.
\item Click on \underline{Import data}, and follow instructions, to
  read in data from a file.  The data set is read into the workspace,
  at the same time becoming the active data set.
\item Click on \underline{New data set \ldots}, then entering data via
  a spreadsheet-like interface.
\item Click on \underline{Data in packages}, then \underline{Read
    Data from Package}.  Then select an attached package and
  choose a data set from among those included with the
  package.
\item A further possibility is to load data from an R image
  (\path{.RData}) file; click on \underline{Load data set \ldots}
\end{itemize}

\paragraph{Creating graphs:}
To draw graphs, click on the \underline{Graphs} drop-down menu. Then,
among other possibilities:
\begin{itemize}
\item Click on \underline{Scatterplot \ldots} to obtain a
  scatterplot.\sidenote[][-24pt]{This uses \margtt{scatterplot()} (\pkg{car}
    package), which in turn uses functions from base graphics.}
\item Click on \underline{X$\;$Y conditioning plot \ldots}
for \pkg{lattice} scatterplots and panels of scatterplots.
\item Click on \underline{3D graph} to obtain a 3D
scatterplot.\footnote{This uses
  the R Commander function \margtt{scatter3d()} that is an interface
  to functions in the \pkg{rgl} package.}
\end{itemize}

\paragraph{Statistics (\& fitting models):} Click on the
\underline{Statistics} drop down menu to get submenus that give
summary statistics and/or carry out various statistical tests.  This
includes (under \underline{Contingency tables}) tables of counts
and (under \underline{Means}) \underline{One-way ANOVA}.
Also, click here to get access to the \underline{Fit models} submenu.

\paragraph{*Models:} Click here to extract information from model
objects once they have been fitted.  (NB: To fit a model, go to the
\underline{Statistics} drop down menu, and click on \underline{Fit
  models}).

\subsection*{Other GUIs for R}

The \pkg{rattle} GUI, aimed broadly at ``data mining'' (data
manipulation, regression, classification and clustering) applications,
is a powerful and sophisticated system.  It has a number of features
that make it attractive for use in standard data mining applications.
Note also \pkg{JGR} (Java Graphics for R).


