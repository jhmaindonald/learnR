

\marginnote[12pt]{CRAN is the primary R `repository'.  For 
preference, obtain R and R packages from a CRAN mirror in 
the local region, e.g.,\\
  Australia: \url{cran.csiro.au/}\\
  NZ: \url{cran.stat.auckland.ac.nz/}}

\noindent
\fbox{\parbox{\textwidth}{
{\bf Note the following web sites:}\\[4pt]
CRAN (Comprehensive R Archive Network):\newline
\url{http://cran.r-project.org}\\[3pt]
% For Sweden, use \url{http://ftp.sunet.se/pub/lang/CRAN/}\\
% For Denmark, use \url{http://cran.dk.r-project.org/}
The Bioconductor repository 
  (\url{http://www.bioconductor.org}), with packages that cater for
  high throughput genomic data, is one of several
  repositories that supplement CRAN.\\[3pt]
R homepage: \url{http://www.r-project.org/}\\[3pt]
Links to useful web pages: From an R session that uses the
GUI, click on \underline{Help}, then on \underline{R Help}. On the 
window that pops up, click on \underline{Resources}
}}

%
%
\section*{Commentary on R}

\subsection*{General}
R has extensive \marginnote{R is free to download from a CRAN site
  (see above).  It runs on all common types of system -- Windows, Mac,
  Unix and Linux.}  graphical abilities that are tightly linked with
its analytic abilities. A new release of base R, on which everything
else is built appears every few months.

The major part of R's abilities for statistical analysis
and for specialist graphics comes from the extensive enhancements that
the packages build on top of the base system.  Its abilities are
further extended by an extensive range of interfaces into
other systems\sidenote{These include Python, SQL and other
  databases, parallel computing using MPI, and Excel.}

The main part of the R system -- base R plus the recommended packages
-- is under continuing development.
\subsection*{The R user base}

Statistical and allied professionals\marginnote{ The R Task Views web
  page (\url{http://cran.csiro.au/web/views/}) notes, for application
  areas where R is widely used, relevant packages.  } who wish to
develop or require access to cutting edge tools find R especially
attractive.  It is finding use, also, as an environment in which to
embed applications whose primary focus is not data analysis or
graphics.

\subsection*{Getting help}

\begin{fullwidth}
\fbox{\parbox{1.15\textwidth}{
{\bf Note the web sites:}\\[4pt]
Wikipedia:\newline
\url{http://en.wikipedia.org/wiki/R_(programming_language)}\\[3pt]
R-downunder (low traffic, friendly):\newline
\url{https://list.science.auckland.ac.nz/sympa/info/stat-rdownunder}\\[3pt]
Stackoverflow\newline
\url{http://stackoverflow.com/questions/tagged/r}.
}}
\end{fullwidth}

The r-help mailing list\marginnote{Details of this and other lists can
  be found at: \url{http://www.r-project.org}. Be sure to check the
  available documentation before posting to r-help. List archives can
  be searched for previous questions and answers.}  serves, especially
for users with a technical bent, as an informal support network.  The
R community expects users to be serious about data analysis, to want
more than a quick cook-book fix, and to show a willingness to work at
improving statistical knowledge.

Novices may find the low traffic R-downunder list more accessible
than the main R mailing list. Subscribers include some highly 
expert users.

\subsection*{The origins and future of R}

The R system implements a dialect of the S language \marginnote{Open
  source systems that might have been the basis for an R-like project
  include Scilab, Octave, Gauss, Python, Lisp-Stat and now Julia.
  None of these can match the range and depth of R's packages,
  with new packages building on what is already there.  Julia's
potential has still to be tested.}
that was developed at AT\&T Bell Laboratories for use as
a general purpose scientific language, but with especial strengths in
data manipulation, graphical presentation and statistical
analysis. The commercial S-PLUS implementation of S popularized the S
language, giving it a large user base of statistical professionals and
skilled scientific users into which R could tap.

Ross Ihaka and Robert Gentleman, both at that time from the University
of Auckland, developed the initial version of R, for use in teaching.
Since mid-1997, development has been overseen by a `core team'
of about a dozen people, drawn from different institutions worldwide.

\marginnote[12pt]{More than 6000 packages are now available through
  the CRAN sites.}
With the release of version 1.0 in early 2000, R became a serious tool
for professional use.  Since that time, the pace of development has
been frenetic, with a new package appearing every week or two.

\marginnote[12pt]{R code looks at first glance like C code. The R
interpreter is modeled on the Scheme LISP dialect.}
The R system uses a language model that dates from the
1980s.  Any change to a more modern language model is likely to be
evolutionary.  Details of the underlying computer implementation will
inevitably change, perhaps at some point radically. Among
more recent language systems \marginnote{Julia strongly
  outperforms R in execution time comparisons that appear on the Julia
  website.}  that have the potential to provide R-like functionality,
Julia (\url{http://julialang.org}) seems particularly interesting.

  \subsection*{Interactive development environments -- editors and more}
  \marginnote{Note also Emacs, with the ESS (Emacs Speaks Statistics)
    addon. is This is a feature-rich environment that can be daunting
    for novices.  It runs on Windows as well as Linux/Unix and Mac.
    Note also, for Windows, the Tinn-R editor
    (\url{http://www.sciviews.org/Tinn-R/}).}  RStudio
  (\url{http://rstudio.org/}) is a very attractive run-time
  environment for R, available for Windows, Mac and Linux/Unix
  systems.  This has extensive abilities for managing projects, and
  for working with code.  It is a highly recommended alternative to
  the GUIs that come with the Windows and Mac OS X binaries that are
  available from CRAN sites.

\subsection*{Pervasive unifying ideas}
Ideas that pervade R include:\\[-8pt]
\marginnote{Expressions can be:\\
\vspace*{-8pt}
\begin{list}{}{\setlength{\itemsep}{2pt} \setlength{\parsep}{0pt}}
\item[] evaluated (of course)

\item[] printed on a graph (come to think of it, why not?)
\end{list}

\noindent There are many unifying computational features.  Thus
any `linear' model (lm, lme, etc) can use spline basis
  functions to fit spline terms.
}
\begin{list}{}{\setlength{\itemsep}{1pt} \setlength{\parsep}{1pt}}
\item[] Generic functions for common tasks -- print, summary, plot, etc.
(the Object-oriented idea; do what that ``class'' of object requires)

\item[] Formulae, for specifying graphs, models and tables.

\item[] Language structures can be manipulated, just like any
data object (Manipulate formulae, expressions, function argument
lists, \dots)

\item[] Lattice (trellis) and ggplot graphics offer innovative
  features that are widely used in R packages.  They aid the provision
  of graphs that reflect important aspects of data structure.

\end{list}
Note however that these are not uniformly implemented through R.
This reflects the incremental manner in which R has developed.

\subsection*{Data set size}
R's evolving technical design has allowed it,\marginnote{An important
  step was the move, with the release of version 1.2, to a dynamic
  memory model.} taking advantage of advances in computing hardware,
to steadily improve its handling of large data sets. The flexibility
of R's memory model does however have a cost\sidenote{The difference
  in cost may be small or non-existent for systems that have a 64-bit
  address space.} for some large computations, relative to systems
that process data from file to file.

\subsection*{Good planning,  informed analysis and reliable software}

While the R system
\marginnote{Take particular care with
  newer or little-used abilities in contributed packages.  These may
  not have been much tested, unless by their developers.  The
greatest risks arise from inadequate understanding of the statistical
issues.}
is unique in the extent of close scrutiny that it
receives from highly expert users, the same warnings apply as to any
statistical system.  The base system and the recommended
  packages get unusually careful scrutiny.

  The scientific context, has crucial implications for the experiments
  that it is useful to do, and for the analyses that are meaningful.
  Available statistical methodology, and statistical and computing
  software and hardware, bring their own constraints and opportunities.

\textit{Statistics of data collection}\marginnote{The same general
  issues arise in field, industrial, medical, biological and
  laboratory experimentation. The aim is, always, is to get maximum
  value from resources used.}  encompasses statistical
\textit{experimental design}, sampling design, and more besides.
Planning will be most effective if based on sound knowledge of the
materials and procedures available to experimenters.

Once the data have been collected, the challenges are then those of
data analysis and of interpretation and presentation of results. For
this, software that is of high quality must be complemented with the
critical resources of well-trained and well-informed minds.

\section*{Documentation and Learning Aids}
\paragraph{R podcasts:} See for example
\url{http://www.r-podcast.org/}

\paragraph{Official Documentation:}
Users who are working through these notes on their own should
have available for reference the document
``An Introduction to R'', written by the R Development Core Team.
To download an up-to-date copy, go to CRAN.

\paragraph{Web-based Documentation:}

* See Computerworld's list (July 11 2018) of "Top R language
resources to improve your data skills":  
\url{https://bit.ly/2SPN0gq}
* Go to \url{http://www.r-project.org}\marginnote{Also
  \url{http://wiki.r-project.org/rwiki/doku.php}}
and look under \underline{Documentation}.
There are further useful links under \underline{Other}.

\paragraph{The R Journal (formerly R News):}
Successive issues are a mine of useful information.
These can be copied down from a CRAN site.

\paragraph{Books:}
See \url{http://www.R-project.org/doc/bib/R.bib} for a list of
R-related books that is updated regularly. Here, note
especially:\\[3pt]
\noindent
Maindonald, J. H. \& Braun, J. H. Data Analysis \&
  Graphics Using R. An Example-Based Approach. 3$^{rd}$ edn, Cambridge
  University Press,
  Cambridge, UK, 2010.\\
\noindent
\url{http://www.maths-people.anu.edu.au/~johnm/r-book.html}

\cleardoublepage
\section*{Notes for Readers of this Text}

\subsection*{Asterisked Sections or Subsections}

Asterisks are used to identify material that is more technical or
specialized, and that might be omitted at a first reading.

\subsection*{The {\em DAAGviz} package}

\marginnote[8pt]{Note that {\em DAAGviz} is not, currently at least,
available on CRAN.}
\marginnote[6pt]{Note that {\em DAAGviz} is not available on CRAN.
It collects scripts and datasets together in a way that may be
useful to course participants.  Those materials are also
available separately.}
>>>>>>> e5442844fdb783da5761d94f5becf5cd061e2bd0
This package is an optional companion to these notes. 
It collects scripts and datasets together in a way that may be
useful to users of this text.  
You can install it, assuming a live internet connection,
by typing:

\begin{fullwidth}

\begin{Schunk}
\begin{Sinput}
remotes::install_github('jhmaindonald/DAAGviz',
                        build_opts=c("--no-resave-data", 
                                     "--no-manual"))
\end{Sinput}
\end{Schunk}

\end{fullwidth}

Assuming that the {\em DAAGviz} package has been installed, it can be attached thus:
\begin{Schunk}
\begin{Sinput}
library(DAAGviz)
\end{Sinput}
\end{Schunk}

Once attached, this package gives access to:
\begin{itemizz}
\item[-]
Scripts that include all the code. To access these scripts do, e.g.
\begin{marginfigure}[66pt]
More succinctly, use the function \margtt{getScript()}:\\[-3pt]
\begin{Schunk}
\begin{Sinput}
## Place Ch 5 script in
## working directory
getScript(5)
\end{Sinput}
\end{Schunk}
\end{marginfigure}
\begin{Schunk}
\begin{Sinput}
## Check available scripts
dir(system.file('scripts', package='DAAGviz'))
## Show chapter 5 script
script5 <- system.file('scripts/5examples-code.R',
                       package='DAAGviz')
file.show(script5)
\end{Sinput}
\end{Schunk}
\item[-]
Source files (also scripts) for functions that can be used to
  reproduce the graphs. These are available for Chapters 5 to 15
only.  To load the Chapter 5 functions into the workspace,
use the command:
\begin{marginfigure}[54pt]
More succinctly, use the function \margtt{sourceFigFuns()}:\\[-3pt]
\begin{Schunk}
\begin{Sinput}
## Load Ch 5 functions
## into workspace
sourceFigFuns(5)
\end{Sinput}
\end{Schunk}
\end{marginfigure}
\begin{Schunk}
\begin{Sinput}
path2figs5 <- system.file('doc/figs5.R',
                          package='DAAGviz')
source(path2figs5)
\end{Sinput}
\end{Schunk}
\item[-] The datasets \code{bronchit}, \code{eyeAmp}, and
  \code{Crimean}, which feature later in these notes.
\end{itemizz}
\newpage

\subsection*{Alternative sources for some datasets}
\marginnote{At courses where these notes are used, these will
be provided on a memory stick.}
{\color{gray40}
The web page \url{http://maths-people.anu.edu.au/~johnm/} may in a few cases
be a convenient source for datasets that are referred to in this text:
\begin{itemizz}
\item[-] 
Look in \url{http://maths-people.anu.edu.au/~johnm/r/rda} for
  various image ({\bf .RData}) files.\sidenote{{\em Image} files hold 
  copies of one or more R objects, in a format that facilitates rapid 
  access from R.}  Use the function \code{load()} to bring any of these 
  into R.
\item[-] Look in \url{http://maths-people.anu.edu.au/~johnm/datasets/text}
  for the files {\bf bestTimes.txt}, {\bf molclock.txt}, and other such
  text files.
\item[=] Look in \url{http://maths-people.anu.edu.au/~johnm/datasets/csv}
  for several {\bf .csv} files.
\end{itemizz}
\vspace*{-9pt}
}

