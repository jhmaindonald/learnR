

\begin{fullwidth}
This is an introduction to R that emphasises statistical applications.
It provides relatively terse overviews, with worked examples, of a
wide range of the abilities that are available from R and from R
packages.  The R system has been in the front line of the huge
advances in software for data analysis from around the end of the
last century.

A strong emphasis is on good research practice.  Always, whether
or not one chooses to use the word `science', what is in mind is
the used of defendable scientific processes.  Results from the
processing of data through a series of analysis steps achieves
nothing, unless there is some wider context of understanding that
gives it meaning.  What do we hope to learn?  What are the processes
that generated the data?  Is there a wider context (typically, there
is) to which we hope that results will apply?

Any use of data that extends beyond providing statistical or
graphical summary relies on the use of a mathematical model,
whether implicitly or explicitly.  Models must be able to
survive reasonable challenges to their use and relevance to
the task in hand, if claims that result from their use are
to be credible.  What is being called the `reproducibility
crisis' in science is in large part a consequence of the neglect
of this key point.

The scientific context has crucial implications for the experiments
that it is useful to do, and for the analyses that are meaningful.
Available statistical methodology, and statistical and computing
software and hardware, bring their own constraints and opportunities.
Effective data analysis requires the critical resources of
well-trained and well-informed minds, with software that is of high 
quality handling the needed calculations.

\subsection*{Good planning, and reliable software}

The \textit{Statistics of data collection}  encompasses statistical
\textit{experimental design}, sampling design, and more besides.
Planning will be most effective if based on sound knowledge of the
materials and procedures available to experimenters.
The same general issues arise in all fields --- industrial, medical, 
biological and laboratory experimentation more generally. The aim 
is, always, is to get maximum value from resources used.

While the R system is unique in the extent of close scrutiny that it
receives from highly expert users, the same warnings apply as to any
statistical system.  The base system and the recommended
  packages get unusually careful scrutiny.  Take particular care 
  with newer or little-used abilities 
in contributed packages.  These may not have been much tested, 
unless by their developers.  The greatest risks arise from 
inadequate understanding of the statistical issues.

Statisticians who work with other scientists, as the author has
for most of a long career, have the satisfaction, and challenge, of
learning about the areas of science to which they have been exposed.
In the author's case, that experience has ranged widely.  It is
reflected in the wide-ranging examples that are presented in the
pages that follow.
\end{fullwidth}

\newpage
\section*{Important R resources}
\marginnote[12pt]{CRAN is the primary R `repository'.  For 
preference, obtain R and R packages from a CRAN mirror in 
the local region, e.g.,\\
  Australia: \url{cran.csiro.au/}\\
  NZ: \url{cran.stat.auckland.ac.nz/}}

\noindent
\fbox{\parbox{\textwidth}{
{\bf Note the following web sites:}\\[4pt]
CRAN (Comprehensive R Archive Network):\newline
\url{http://cran.r-project.org}\\[3pt]
% For Sweden, use \url{http://ftp.sunet.se/pub/lang/CRAN/}\\
% For Denmark, use \url{http://cran.dk.r-project.org/}
The Bioconductor repository 
  (\url{http://www.bioconductor.org}), with packages that cater for
  high throughput genomic data, is one of several
  repositories that supplement CRAN.\\[3pt]
R homepage: \url{http://www.r-project.org/}\\[3pt]
Links to useful web pages: From an R session that uses the
GUI, click on \underline{Help}, then on \underline{R Help}. On the 
window that pops up, click on \underline{Resources}
}}


\subsection*{The {\em DAAGviz} package}

\marginnote[8pt]{Note that {\em DAAGviz} is not, currently at least,
available on CRAN.}
\marginnote[6pt]{Note that {\em DAAGviz} is not available on CRAN.
It collects scripts and datasets together in a way that may be
useful to course participants.  Those materials are also
available separately.}
>>>>>>> e5442844fdb783da5761d94f5becf5cd061e2bd0
This package is an optional companion to these notes. 
It collects scripts and datasets together in a way that may be
useful to users of this text.  
You can install it, assuming a live internet connection,
by typing:

\begin{fullwidth}

\begin{Schunk}
\begin{Sinput}
remotes::install_github('jhmaindonald/DAAGviz',
                        build_opts=c("--no-resave-data", 
                                     "--no-manual"))
\end{Sinput}
\end{Schunk}

\end{fullwidth}

Assuming that the {\em DAAGviz} package has been installed, it can be attached thus:
\begin{Schunk}
\begin{Sinput}
library(DAAGviz)
\end{Sinput}
\end{Schunk}

Once attached, this package gives access to:
\begin{itemizz}
\item[-]
Scripts that include all the code. To access these scripts do, e.g.
\begin{marginfigure}[66pt]
More succinctly, use the function \margtt{getScript()}:\\[-3pt]
\begin{Schunk}
\begin{Sinput}
## Place Ch 5 script in
## working directory
getScript(5)
\end{Sinput}
\end{Schunk}
\end{marginfigure}
\begin{Schunk}
\begin{Sinput}
## Check available scripts
dir(system.file('scripts', package='DAAGviz'))
## Show chapter 5 script
script5 <- system.file('scripts/5examples-code.R',
                       package='DAAGviz')
file.show(script5)
\end{Sinput}
\end{Schunk}
\item[-]
Source files (also scripts) for functions that can be used to
  reproduce the graphs. These are available for Chapters 5 to 15
only.  To load the Chapter 5 functions into the workspace,
use the command:
\begin{marginfigure}[54pt]
More succinctly, use the function \margtt{sourceFigFuns()}:\\[-3pt]
\begin{Schunk}
\begin{Sinput}
## Load Ch 5 functions
## into workspace
sourceFigFuns(5)
\end{Sinput}
\end{Schunk}
\end{marginfigure}
\begin{Schunk}
\begin{Sinput}
path2figs5 <- system.file('doc/figs5.R',
                          package='DAAGviz')
source(path2figs5)
\end{Sinput}
\end{Schunk}
\item[-] The datasets \code{bronchit}, \code{eyeAmp}, and
  \code{Crimean}, which feature later in these notes.
\end{itemizz}
\newpage

\subsection*{Alternative sources for some datasets}
\marginnote{At courses where these notes are used, these will
be provided on a memory stick.}
{\color{gray40}
The web page \url{http://maths-people.anu.edu.au/~johnm/} may in a few cases
be a convenient source for datasets that are referred to in this text:
\begin{itemizz}
\item[-] 
Look in \url{http://maths-people.anu.edu.au/~johnm/r/rda} for
  various image ({\bf .RData}) files.\sidenote{{\em Image} files hold 
  copies of one or more R objects, in a format that facilitates rapid 
  access from R.}  Use the function \code{load()} to bring any of these 
  into R.
\item[-] Look in \url{http://maths-people.anu.edu.au/~johnm/datasets/text}
  for the files {\bf bestTimes.txt}, {\bf molclock.txt}, and other such
  text files.
\item[=] Look in \url{http://maths-people.anu.edu.au/~johnm/datasets/csv}
  for several {\bf .csv} files.
\end{itemizz}
\vspace*{-9pt}
}

\subsection*{Asterisked Sections or Subsections}

Asterisks are used to identify material that is more technical or
specialized, and that might be omitted at a first reading.

