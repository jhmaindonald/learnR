% !Rnw root = learnR.Rnw




\section*{The scope of these notes}

These notes were developed over the course of more than a
decade, for use in R courses that were presented to groups
within universities, within CSIRO, and within Government
departments.  Each new course offered the chance to 
extend and refine the content, and to add content that
was tuned to the requirements of the new audience.  
The result is a somewhat eclectic mix of material.
The notes are provided here with the intention that
others will be free to add to them, or to develop them
for their own purposes.
%
%
\section*{Commentary on R}

\subsection*{General}
R has extensive \marginnote{R is free to download from a CRAN site
  (see above).  It runs on all common types of system -- Windows, Mac,
  Unix and Linux.}  graphical abilities that are tightly linked with
its analytic abilities. A new release of base R, on which everything
else is built appears every few months.

The major part of R's abilities for statistical analysis
and for specialist graphics comes from the extensive enhancements that
the packages build on top of the base system.  Its abilities are
further extended by an extensive range of interfaces into
other systems\sidenote{These include Python, SQL and other
  databases, parallel computing using MPI, and Excel.}

The main part of the R system -- base R plus the recommended packages
-- is under continuing development.

\subsection*{The R user base}

Statistical and allied professionals\marginnote{ The R Task Views web
  page (\url{http://cran.csiro.au/web/views/}) notes, for application
  areas where R is widely used, relevant packages.} have found R
  especially attractive, both for the access that it gives to
  cutting edge tools, and as a platform for developing new tools.
Additionally, the R system has found wide use among working
scientists whose data analysis requirements justify the time
needed to gain the necessary R skills.  It is finding use, also, as
an environment in which to embed applications whose primary focus
is not data analysis or graphics.


\subsection*{Getting help}

\begin{fullwidth}
\fbox{\parbox{1.15\textwidth}{
{\bf Note the web sites:}\\[4pt]
Wikipedia:\newline
\url{http://en.wikipedia.org/wiki/R_(programming_language)}\\[3pt]
R-downunder (low traffic, friendly):\newline
\url{http://www.stat.auckland.ac.nz/mailman/listinfo/r-downunder}\\[3pt]
Stackoverflow\newline
\url{http://stackoverflow.com/questions/tagged/r}.
}}
\end{fullwidth}
\vspace*{8pt}

The r-help mailing list\marginnote{Details of this and other lists can
  be found at: \url{http://www.r-project.org}. Be sure to check the
  available documentation before posting to r-help. List archives can
  be searched for previous questions and answers.}  serves, especially
for users with a technical bent, as an informal support network.  The
R community expects users to want more than a quick cook-book fix, and
to show a willingness to work at improving statistical knowledge.

Novices will find the low traffic R-downunder list more friendly and
helpful than the main R mailing list. Its subscribers include some
highly expert individuals.

\subsection*{Important R web links}

\marginnote{CRAN is the primary R `repository'.  Among
  package repositories that supplement CRAN, note in
  particular the Bioconductor repository
  (\url{http://www.bioconductor.org}), which caters for
  high throughput genomic data.}
\noindent
\fbox{\parbox{\textwidth}{
{\bf Note the following web sites:}\\[4pt]
CRAN (Comprehensive R Archive Network):\newline
\url{http://cran.r-project.org}\\[3pt]
Obtain R and R packages from a CRAN mirror in the local region.
An Australian mirror (one of two) is: \url{https://cran.csiro.au/}\\[3pt]
For New Zealand, use \url{http://cran.stat.auckland.ac.nz}\\
R homepage: \url{https://www.r-project.org/}\\[3pt]

For various useful links click, from an R session that uses the
GUI, on the menu item \underline{R help}. Then, on the browser window
that pops up, look under \underline{Resources}}}



\subsection*{The origins and future of R}

The R system implements a dialect of the S language \marginnote{Open
  source systems that might have been the basis for an R-like project
  include Scilab, Octave, Gauss, Lisp-Stat, Python, and now Julia.
  None of these match the range and depth of R's packages.  
  Julia, which strongly outperforms R in execution time comparisons
 that appear on the Julia website \url{http://julialang.org},
 has not had time to establish a clear place for itself.}
that was developed at AT\&T Bell Laboratories as a general purpose
scientific language, with especial strengths in data manipulation,
graphical presentation and statistical analysis. The commercial
S-PLUS implementation popularized the S language, giving a user
base into which R could tap.

Ross Ihaka and Robert Gentleman, both at that time from the University
of Auckland, developed the initial version of R, for use in teaching.
Since mid-1997, development has been overseen by a `core team'
of about a dozen people, drawn from different institutions worldwide.

\marginnote[12pt]{More than 12,000 packages are, as of January 2018,
available through the CRAN sites.}
With the release of version 1.0 in early 2000, R became a serious tool
for professional use.  Since 2004, the number of packages has
increased at a rate of slightly more than 25\% per annum.

\marginnote[12pt]{R code looks at first glance like C code. The R
interpreter is modeled on the Scheme LISP dialect.}
The R system uses a language model that dates from the 1980s.
Any change to a more modern language model is likely to be
evolutionary.  Details of the underlying computer implementation
are in a process of limited continual change. 

  \subsection*{Interactive development environments -- editors and more}
  \marginnote{Note also Emacs, with the ESS (Emacs Speaks Statistics)
    addon. is This is a feature-rich environment that can be daunting
    for novices.  It runs on Windows as well as Linux/Unix and Mac.
    Note also, for Windows, the Tinn-R editor
    (\url{http://www.sciviews.org/Tinn-R/}).}  RStudio
  (\url{http://rstudio.org/}) is a very attractive run-time
  environment for R, available for Windows, Mac and Linux/Unix
  systems.  This has extensive abilities for managing projects, and
  for working with code.  It is a highly recommended alternative to
  the GUIs that come with the Windows and Mac OS X binaries that are
  available from CRAN sites.

\subsection*{Pervasive unifying ideas}
Ideas that pervade R include:\\[-8pt]
\marginnote{Expressions can be:\\
\vspace*{-8pt}
\begin{list}{}{\setlength{\itemsep}{2pt} \setlength{\parsep}{0pt}}
\item[] evaluated (of course)

\item[] printed on a graph (come to think of it, why not?)
\end{list}

\noindent There are many unifying computational features.  Thus
any `linear' model (lm, lme, etc) can use spline basis
  functions to fit spline terms.
}
\begin{list}{}{\setlength{\itemsep}{1pt} \setlength{\parsep}{1pt}}
\item[] Generic functions for common tasks -- print, summary, plot, etc.
(the Object-oriented idea; do what that ``class'' of object requires)

\item[] Formulae, for specifying graphs, models and tables.

\item[] Language structures can be manipulated, just like any
data object (Manipulate formulae, expressions, function argument
lists, \dots)

\item[] Lattice (trellis) and ggplot graphics offer innovative
  features that are widely used in R packages.  They aid the provision
  of graphs that reflect important aspects of data structure.

\end{list}
Note however that these are not uniformly implemented through R.
This reflects the incremental manner in which R has developed.

\subsection*{Data set size}
R's evolving technical design has allowed it,\marginnote{An important
  step was the move, with the release of version 1.2, to a dynamic
  memory model.} taking advantage of advances in computing hardware,
to steadily improve its handling of large data sets. The flexibility
of R's memory model does however have a cost\sidenote{The difference
  in cost may be small or non-existent for systems that have a 64-bit
  address space.} for some large computations, relative to systems
that process data from file to file.

\subsection*{Good planning,  informed analysis and reliable software}

While the R system
\marginnote{Take particular care with newer or little-used abilities
in contributed packages.  These may not have been much tested,
unless by their developers.} is unique in the extent of close
scrutiny that it receives from highly expert users, the same
warnings apply as to any statistical system.  The base system and
the recommended packages get unusually careful scrutiny.

  The scientific context, has crucial implications for the experiments
  that it is useful to do, and for the analyses that are meaningful.
  Available statistical methodology, and statistical and computing
  software and hardware, bring their own constraints and opportunities.

\textit{Statistics of data collection}\marginnote{Always, one
has to ask whether data are available, or can be collected, 
that allow the required inferences.} encompasses statistical
\textit{experimental design}, sampling design, and data collection
more generally. Subject area insights can be crucial. 

Once the data have been collected, the challenges are then those of
data analysis and of interpretation and presentation of results. 
Effective data analysis must take account of the limitations
inherent in the data, an understanding of the statistical issues,
and 
risks that arise from inadequate understanding of the statistical
issues.
For
this, software that is of high quality must be complemented with the
critical resources of well-trained and well-informed minds.

\section*{Documentation and Learning Aids}
\paragraph{R podcasts:} See for example
\url{http://www.r-podcast.org/}

\paragraph{Official Documentation:}
Users who are working through these notes on their own should
have available for reference the document
``An Introduction to R'', written by the R Development Core Team.
To download an up-to-date copy, go to CRAN.

\paragraph{Web-based Documentation:}

Go to \url{http://www.r-project.org}\marginnote{Also
  \url{http://wiki.r-project.org/rwiki/doku.php}}
and look under \underline{Documentation}.
There are further useful links under \underline{Other}.

\paragraph{The R Journal (formerly R News):}
Successive issues are a mine of useful information.
These can be copied down from a CRAN site.

\paragraph{Books:}
See \url{http://www.R-project.org/doc/bib/R.bib} for a list of
R-related books that is updated regularly. Here, note
especially:\\[3pt]
\noindent
Maindonald, J. H. \& Braun, J. H. 2010. Data Analysis \&
  Graphics Using R. An Example-Based Approach. 3$^{rd}$ edn, Cambridge
  University Press,
  Cambridge, UK, 2010.\\
\noindent
\url{http://www.maths.anu.edu.au/~johnm/r-book.html}

\cleardoublepage
\section*{Notes for Readers of this Text}

\subsection*{Asterisked Sections or Subsections}

Asterisks are used to identify material that is more technical or
specialized, and that might be omitted at a first reading.

\subsection*{The {\em DAAGviz} package}
\marginnote{The {\em DAAGviz} package collects scripts and
datasets together in a way that may be useful to readers of
these notes.}
This package, available from Github, is an optional companion
to these notes. 

\begin{marginfigure}Assuming that the {\em DAAGviz} package has been installed, it can be attached thus:
\begin{Schunk}
\begin{Sinput}
library(DAAGviz)
\end{Sinput}
\end{Schunk}
\end{marginfigure}
Once attached, this package gives access to:
\begin{itemizz}
\item[-]
\begin{marginfigure}[78pt]
More succinctly, use the function \margtt{getScript()}:\\[-3pt]
\begin{Schunk}
\begin{Sinput}
## Place Ch 5 script in
## working directory
getScript(5)
\end{Sinput}
\end{Schunk}
\end{marginfigure}
Scripts that include all the code. To access these scripts do, e.g.
\begin{Schunk}
\begin{Sinput}
## Check available scripts
dir(system.file('scripts', package='DAAGviz'))
## Show chapter 5 script
script5 <- system.file('scripts/5examples-code.R',
                        package='DAAGviz')
file.show(script5)
\end{Sinput}
\end{Schunk}
\item[-]
\begin{marginfigure}[24pt]
More succinctly, use the function \margtt{sourceFigFuns()}:\\[-3pt]
\begin{Schunk}
\begin{Sinput}
## Load Ch 5 functions
## into workspace
sourceFigFuns(5)
\end{Sinput}
\end{Schunk}
\end{marginfigure}
Source files (also scripts) for functions that can be used to
  reproduce the graphs. These are available for Chapters 5 to 15
only.  To load the Chapter 5 functions into the workspace,
use the command:
\begin{Schunk}
\begin{Sinput}
path2figs5 <- system.file('doc/figs5.R',
                          package='DAAGviz')
source(path2figs5)
\end{Sinput}
\end{Schunk}
\item[-] The datasets \txtt{bronchit}, \txtt{eyeAmp}, and
  \txtt{Crimean}, which feature later in these notes.
\end{itemizz}
\newpage
